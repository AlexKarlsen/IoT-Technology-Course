\subsection*{Design Alternatives}
Here we discuss design alternatives to the solution. In scenarios with strict deterministic real-time requirements, micro controllers with deterministic guarantees would be preferable. RPis are actually used in real-life IoT applications and there exist a version, Revolution Pi \cite{revolution}, that adhere to industry standards. Even though the Raspberry Pis are rather cheap, it would in some cases be cheaper to use micro controllers.
The device could be used as an internet gateway for a wireless sensor network (WSN). The WSN could be a non-internet connected network e.g. ZigBee network, these more low-level devices could expand the monitoring range and precision with minimal cost. 
There exist different options for communication protocols, MQTT is one solution, whereas AMQP or HTTP are alternatives. HTTP however is not nearly as lightweight and it is document centric instead of data centric. MQTT are chosen as it is recommended in the project description and taught in the course. A comparison of MQTT and AMQP would be interesting, but is considered out of project's scope.
Web sockets could have been preferred over SSE. It enabled bidirectional communication between server and clients as opposed to SSE one-way communication. Web sockets enabled a more subscribable pattern, where different clients may only be interested in specific data sources. It removes the need for the client to filter the data send to the EventSource interface.
There exists a wide range of technologies and frameworks for building web applications. We have chosen not to bring any alternatives to Angular. Angular has been chosen due to prior experience with the framework. 
