\subsection{SSE}
Stream data capabilities are important to be able for real-time monitoring of data in IoT systems. Additionally alarms will be raised on faulty conditions or on error in the system, which requires human actors or other computer system to be notified. Different options exist to obtain real-time data streaming e.g. Long Polling, Server-Sent Events (SSE) and Web Sockets. SSE is an API for opening HTTP connection for receiving push notifications. SSE has been chosen for simplicity, as it is based on HTTP, and do not introduce network overhead. The downside of using SSE is the simple text-based approach, which requires the client to filter the streamed messages \cite{evalSSE}. The SSE recommendation \cite{W3C} is developed by the World Wide Web Consortium (W3C). The open HTTP connection allows the server to continuously write messages to the client. Client browsers must construct the EventSource interface and can receive message from the onmessage callback function.