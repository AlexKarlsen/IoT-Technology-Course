\subsection{Digital Twin}
IoT applications often have real-time requirements, not necessarily timely but with low latency. With many hundreds of connected IoT devices, it is an issue to manage the state. Major players in the IoT market have introduces the digital twin to their IoT platform. This implements and examines the digital twin. The digital twin is a virtual; "twin", "copy", "shadow" or "doppelganger" of a physical entity. Roberto Sarracco former head of EIT Digital Industrial Doctoral School, mentions in a blog post on IEEE's website, that General Electronics probably where the first to come up with the concept of the digital twin about 15 years ago \cite{IEEE}.
General Electronics was one of the first to see the potential of the digital twin in manufacturing. They saw the potential of using the digital twin in simulations to optimize the product virtually before making any expensive prototypes \cite{GE}. The digital twin can be used in the entire life cycle of the product in design, manufacturing and maintenance. The increasing amount of sensor data acquired are used to optimize the processes and the product itself with the use of Big Data Analysis \cite{8477101}. Since then the concept of digital twin has evolved rapidly. It has been a on "Gartner Top 10 Strategic Technology Trends" in both 2017 and 2018 \cite{Gartner}. A a lot of other firms in other industries has adopted the concept, especially within the IoT domain.

Microsoft Azure and Amazon uses the same concept to manage real-time state information of the low-level IoT Devices. Microsoft uses the term "Device Twin" \cite{MS} which a lot of other vendors have taken on e.g. IBM. Amazon has named their edition of the digtial twin for "Device Shadow" \cite{Amazon}. Commonly both are JSON structures describing the real-time state of IoT Devices. The two vendors have implemented the digital twin in their IoT platforms; Microsoft Azure and Amazon Web Services, where a device is linked to its device twin. The twin can be manipulated through a Rest API and message exchange between the IoT platform and device is done trough the MQTT.

Microsoft introduced in \cite{azuredigitaltwin} in september 2018 a new addition to Azure - the Azure Digital Twin. The Azure Digital Twin is a high-level virtual model of a physical setting. They are moving from a thing-centric approach, as the device twin to a high-level domain-centric approach. These virtual domain-level models could be entire buildings, large ship or even entire cities. However these models require a vast amount of low-level devices, that acquires the data for the domain model. We will be looking into the low-level thing-centric device twin approach, but the evolvement in industries was found worth mentioning. 