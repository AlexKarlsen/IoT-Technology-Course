\subsection{Digital Twin}
This project utilizes the emerging concept called Digital Twin. The digital twin is a virtual; "twin", "copy", "shadow" or "doppelganger" of a physical entity. Roberto Sarracco former head of EIT Digital Industrial Doctoral School, mentions in a blog post on IEEE's website, that General Electronics probably where the first to come up with the concept of the digital twin \cite{IEEE}.
General Electronics was one of the first to see the potential of the digital twin in manufacturing. They saw the potential of using the digital twin in simulations and optimize the product virtually before making any expensive prototypes and making the development process digital \cite{GE}. Since then the concept of digital twin has evolved rapidly. It has been a regular member of "Gartner Top 10 Strategic Technology Trends" since 2017 and is once more on the list \cite{Gartner}. A a lot of other firms in other industries has adopted the concept, especially within the IoT domain. 

Microsoft and Amazon uses the same concept to manage real-time state information of IoT Devices. Microsoft uses the term "Device Twin" \cite{MS} which a lot of other vendors have taken on e.g. IBM. Amazon has named their edition of the digtial twin for "Device Shadow" \cite{Amazon}. Commonly both are JSON structures describing the real-time state of IoT Devices. The two vendors have implemented the digital twin in their IoT platforms; Microsoft Azure and Amazon Web Services, where a device is linked to its digital twin. The twin can be manipulated using basic Rest APIs and message exchange between the IoT platform and device is done trough the Message Queuing Telemetry Transfer (MQTT) protocol. 