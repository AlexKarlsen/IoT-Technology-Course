\subsection{MQTT}
IoT devices often exists in constrained networks, e.g. limited bandwidth or high latency, or having limited memory or computational power, which raises new specific communication requirements. The Message Queuing Telemetry Transport (MQTT) protocol extends the Messages Oriented Middleware (MOM), IBM WebSphere MQ to IoT devices.\\
The MQTT architecture consist of a broker and connected clients (e.g. IoT devices) in a star-like topology. MQTT is design to be extremely simple and lightweight. It utilizes the publish/subscribe pattern, for asynchronous and highly decoupled systems, that scales very well while using a small header of only two bytes, making it ideal for these limited devices. This decoupling is accomplished by the use of topics, that can be subscribed or published to. Additionally MQTT consists of three QoS levels; at most once, at least once and exactly once, determining the servers effort to deliver messages.\\
The broker, retains messages, thus being able to send old message to new subscribers. It uses a clean session flag, to determine whether a clients subscriptions should be removed upon disconnect, or not. Lastly it utilizes wills, enabling the server to send a message, should an unexpected disconnect occur.

%The Message Queuing Telemetry Transport (MQTT) is an ISO standard protocol, initially by IBM, it is build on top of TCP/IP. It is a messaging protocol that implements the publish-subscribe pattern\cite{}. It is designed to be lightweigth, e.i. having a small code footprint or for limited network bandwidth\cite{}. This makes it ideal for IoT devices, that are often limited \todo{Elaborate}.\\ MQTT consists of three message types; Connect, Disconnect and Publish\todo{Elaborate}.\\ A QoS metric can be defined as; At most once, At least once, and Exactly once. in regards to number of retries on failure and hand shaking\todo{Elaborate}.