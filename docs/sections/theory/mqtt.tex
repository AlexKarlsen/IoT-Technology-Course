\subsection{MQTT}
IoT devices often exists in constrained networks, e.g. limited bandwidth or high latency, or having limited memory or computational power, which raises new specific communication requirements. The Message Queuing Telemetry Transport (MQTT) protocol is an ISO standard \cite{developing_standards_2016}.\\

The MQTT architecture consist of a broker and connected clients (e.g. IoT devices) in a star topology. MQTT is design to be extremely simple and lightweight. It utilizes the publish/subscribe pattern, giving asynchronous and highly decoupled systems, that scales very well, while still using a small header of only two bytes, making it ideal for these constrained devices. This decoupling is accomplished by the use of topics, that can be subscribed or published to. Additionally MQTT consists of three QoS levels; at most once, at least once and exactly once, determining the servers effort to deliver messages\cite{MQTTBook}.\\

The broker, retains messages, thus being able to send old message to new subscribers. It uses a clean session flag, to determine whether a clients subscriptions should be removed upon disconnect, or not. Lastly it utilizes wills, enabling the server to send a message, should an unexpected disconnect occur\cite{MQTTBook}.