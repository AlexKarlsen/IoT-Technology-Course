\subsection{REST}
Serveral IoT setups utilizes Web Services also called Web APIs to get resources, perform common tasks and to deliver acquired data. Web services comes in many forms, a very popular architecture is Representational State Transfer (REST). REST is an architectural model set forth by Roy Thomas Fielding in his doctoral dissertation \cite{Fielding}. Fielding introduced 6 guiding principles; client-server, stateless, cacheable, unifrom interface, layered system and code on demand. These principles allow the web services to scale and serve a growing number of clients by keeping all the state in the request and let client manage the session. The uniform interface allows different kind of client accessing the resources in a uniform manner. The layered system is a key advantage of REST APIs as it encapsulates the business logic. Coding can be done on demand, while adhering to the REST architecture the Web API can be updated. Letting smart objects and IoT devices continue to consume the services the API offers, which essentially means that the devices does not need to be called back for updates and redeployment. \\

Rest APIs connects Smart Objects to IoT devices through web services using Universal Resource Identifier (URI) to identify objects. The objects requested are transferred in the JSON data format, which are a very compact data format well-suited for constrained IoT devices and Smart devices. RESTful Web APIs implements basic HTTP operations (Get, Post, Put and Delete) and as almost any framework today comes with an HTTP API, it makes it easy to consume the services. When new technologies emerge they can easily integrate with the existing solution by using the REST API. REST API allows to add a layer of security by authenticating and authorization of both IoT devices and Smart devices, that uses the services. As the API sits between IoT devices and Smart device applications, no direct connection is established between the two, thus removing a potential security vulnerability.

REST rely on three concepts: 
\begin{description}
    \item[Representation] Data or resources are encoded as representations of the data or resource. The resource temperature can be represented as a decimal number.
    \item[State] All state needed to provide the request must be included in the request. This provide stateless clients and servers. 
    \item[Transfer] The representations can be transferred between servers and clients.
    \cite{smartthings} 
\end{description} 
