\section{Introduction}
This paper is a mandatory mini-project in the Internet of Things Technology course at Aarhus University. The chosen subject is mini-project 10, the weather station. The objective is to build a weather station using a Raspberry Pi equipped with a Sense HAT to obtain environmental sensor data and the the data should be reported to a back end using a Messages-oriented Middleware (MoM). We extend the weather station to a environmental monitor capable of measuring environmental data and real-time detection of thresholds and alarming. Additionally we added a database for persistence of device state information and acquired sensor telemetry data. \\

Monitoring is an interesting application within the IoT domain, especially for industries as it can lead to new business opportunities such as Industry 4.0. Industry 4.0 takes automated manufacturing even further and brings autonomy into the game. The huge amount of sensor data collected to monitor systems will ideally be used by computers to detect patterns and let computers act to optimize the process and proactively predict maintenance and failures. \\

We investigate along with the aforementioned data acquisition, real-time monitoring and reporting, a low-level version of a digital twin, called a device twin. The digital version of the physical device is used for real-time state information. Device twins are useful in scenarios with sleepy devices or harsh communication environment to provide state consistency. \\

We have implemented the environmental monitor with a Raspberry Pi using relevant technologies taught in the course. This includes Message Queue Telemetry Transfer (MQTT) protocol and REST API. Additionally a NoSQL database is used to implement the device twin and a web application has been developed for end-users, to be able to monitor the environment. \\

The rest of the paper is structured as follows; Section \ref{sec:theory} Theory describes the main theories used to implement the environmental monitor; Digital Twin, Message Queue Telemetry Transfer (MQTT) protocol and Representational State Transfer (REST) architecture. Section \ref{sec:architecture} describes the architecture and justifies relevant architectural choices. Section \ref{sec:impl} explains the implementation and the design choices made to obtain the solution. Section \ref{sec:discussion} discusses the design, design alternatives and future work. Section \ref{sec:conclusion} concludes upon the mini-project.

% This section must give a brief motivation and
% introduction to the topic of the project. The section should also give a brief account of the
% results that have been obtained. The end of this section should give a 5-10 line overview of
% how the rest of the report is organized. Try also to emphasize on whould course element
% you are adressing and how you address them (contributions).

% What is the subject of your investigation/implementation?
% What is the expected outcome?

% What is needed (tools, knowledge, components etc.) for success?

% Why is it interesting and for whom (motivation)?

% What can the possible future use of the outcome be?

% How can the expected outcome be verified/tested?

% The internet of things ...

% Course contexts and relevancy

% Related work ...  MS Device Twin or Amazon Device Shadow maybe articles on digital twin 

% Project 10 weather station ... altered more logic on device e.g. thresholds etc.


