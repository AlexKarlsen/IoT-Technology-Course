\section{Introduction}
This paper is a mandatory mini-project in the Internet of Things Technology course at Aarhus University. The chosen subject is mini-project 10, the weather station, however this mini-project extends the weather station to a environmental monitor capable of real-time detection of thresholds and alarming. Monitoring is an interesting application especially for industries can lead to even new opportunities such as Industry 4.0. Industry 4.0 takes automated manufacturing even further and brings autonomy into the game. The huge amount of sensors data collected to monitor systems will in the ideal of Industry 4.0 be used by computers to detect patterns and let computers act to optimize the process and proactively predict maintenance and failures.

The paper investigates monitoring with environmental sensor and implements the concept of a digital twin or device twin in the environmental monitor to obtain real-time state information about the device. Digital twins are useful in scenarios with sleepy devices or harsh communication environment to provide state consistency.

The paper addresses relevant topics of the course namely; Message-Oriented Middleware and Web Services, but also addresses possibilities with real-time communication and state of IoT devices.  The rest of the paper is structured as follows; Section \ref{sec:theory} Theory describes the main theories used to implement the environmental monitor; Digital Twin, Message Queue Telemetry Transfer (MQTT) protocol and Representational State Transfer (REST) architecture. Section \ref{sec:architecture} describes the architecture and justifies relevant architectural choices. Section \ref{sec:impl} explains the implementation and the design choices made to obtain the solution. Section \ref{sec:discussion} discusses the design, design alternatives and future work. Section \ref{sec:conclusion} concludes upon the mini-project.

% This section must give a brief motivation and
% introduction to the topic of the project. The section should also give a brief account of the
% results that have been obtained. The end of this section should give a 5-10 line overview of
% how the rest of the report is organized. Try also to emphasize on whould course element
% you are adressing and how you address them (contributions).

% What is the subject of your investigation/implementation?
% What is the expected outcome?

% What is needed (tools, knowledge, components etc.) for success?

% Why is it interesting and for whom (motivation)?

% What can the possible future use of the outcome be?

% How can the expected outcome be verified/tested?

% The internet of things ...

% Course contexts and relevancy

% Related work ...  MS Device Twin or Amazon Device Shadow maybe articles on digital twin 

% Project 10 weather station ... altered more logic on device e.g. thresholds etc.


