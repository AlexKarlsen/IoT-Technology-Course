\section{Conclusion} \label{sec:conclusion}
The weather station mini-project is successfully implemented with the extension of monitoring capabilities. It is implemented with streaming capabilities for real-time telemetry data and alarming. The threshold is only on a conceptual level and should be extended with upper and lower threshold bounds. The communication between server and devices is successfully establish with an MQTT broker and the server implements a RESTful architectural pattern. It can be concluded, that the device twin is vital when real-time state information is needed and removes consistency pains and communication retries when working in harsh environment. 
The prototype is relevant in IoT applications scenarios where monitoring, automation and eventually autonomy will play a major part in the coming years. Digital twins helps bridging th digital and analogous world, as digital copies and provides new opportunities.