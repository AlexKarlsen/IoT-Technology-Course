\subsection{Device}
The code running on the device (Raspberry Pi 3 B) can be seen as three main parts; Bootup, Sense HAT and MQTT.

\subsubsection*{Boot}
At boot the device automatically starts the program, which initially loads its latests state from its config json file. Then the recent state is fetched from the device twin via the REST API. The local and global are compared, and the device config is updated if needed. The MQTT is started, which will be described later. Lastly the device is set to periodically perform and send readings in a loop. 

\subsubsection*{Senste HAT}
All code related to using the Sense HAT, have been put in its own python file \textit{sense.py}. This file consist of five functions, and is using the libraries sense\_hat, time and threading.
The first two functions are singleReading and readPeriodically. singleReading, reads from all the sensors and returns the values, readPeriodically, uses singleReading, to get a reading and passes the readings to a handler, and initializes an other reading to be performed after a giving period.
\begin{lstlisting}[language=Python, caption=Python reading from sensehat, label={lst:rprfsh}, basicstyle=\scriptsize]
    sense = SenseHat()

    def singleReading():
        return [sense.temp, sense.humidity, sense.pressure]   

    def readPeriodically(periode, handler): 
        handler(singleReading())
        threading.Timer(periode, readPeriodically, [periode, handler]).start()
\end{lstlisting}
The two next functions uses the 8x8 LED matrix display, to print a happy green or sad red smiley, using a matrix containing RGB values.
\begin{lstlisting}[language=Python, caption=example of setting LED matrix, label={lst:rpslm}, basicstyle=\scriptsize]
    def setInbounds():
        ...
        mark = [...]
        sense.set_pixels(mark)
\end{lstlisting}
Lastly it contains an example reading handler (for readPeriodically) that prints the values to the console.

\subsubsection*{MQTT}
All code related to using the MQTT, have also been put in its own python file \textit{mqtthelper.py}. This file holds a MQTT Client object, has six functions, and uses a bunch of libraries, most importantly paho.mqtt.client. The first function is used to initialize the MQTT Client, this is done by creating the MQTT Client using a simple constructor, followed by assigning the other functions as callbacks for MQTT events. Then connection parameters is read from an environment file and used to connect. Lastly it subscribes to get updates regarding its threshold, in both a general and device specific manner.
\begin{lstlisting}[language=Python, caption=Python MQTT setup, label={lst:rpmqtt}, basicstyle=\scriptsize]
    def create():
        mqttc = mqtt.Client()
        mqttc.on_message = on_message
        mqttc.on_connect = on_connect
        mqttc.on_publish = on_publish
        mqttc.on_subscribe = on_subscribe
        # mqttc.on_log = on_log ## Uncomment to enable debug messages
        ## Getting environment variables, for MQTT
        ...
        ## Connect MQTT
        mqttc.username_pw_set(username, password)
        mqttc.connect(host, port)
        ## Start subscribe to MQTT, with QoS level 0
        mqttc.subscribe("threshold")
        mqttc.subscribe(deviceSubscription)
        return mqttc
\end{lstlisting}
The five other functions, are as mentioned above, callback functions, for; on\_message, on\_connect, on\_publish, on\_subscribe and an on\_log for debugging. The most interesting being on\_message, where the others just prints to the console. on\_message loads the received message as json, and then switches on the message topic. On threshold messages it updates its own config, and reports this back to the server.
\begin{lstlisting}[language=Python, caption=MQTT handler examples, label={lst:rpmqtthe}, basicstyle=\scriptsize]
    def on_message(client, obj, msg):
        tmp = json.loads(msg.payload.decode('utf8'))
        if msg.topic == constants.topics["Threshold"]:
            print(msg.topic)
            config.setThresholdValue(tmp["type"], tmp["value"])
            deviceConfig = config.getSavedState()
        elif msg.topic == deviceSubscription:
            config.setThresholdValue(tmp["type"], tmp["threshold"])
            config.setThresholdValue(tmp["type"], tmp["value"])
            deviceConfig = config.getSavedState()
            report = config.getSavedState()
            reportedState = {
                "msgType": "Report",
                "DeviceId": deviceId,
                "ReportedState": report['ReportedState']
            }
            mqttc.publish("report", json.dumps(reportedState))
        else:
            print('No subscription handler')

    def on_publish(client, obj, mid):
        print("mid: " + str(mid))
\end{lstlisting}