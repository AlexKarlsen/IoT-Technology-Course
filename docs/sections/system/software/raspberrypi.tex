\subsection{Device (Raspberry Pi)}
The Raspberry Pi is equipped with a Sense HAT. When the Pi boots it starts executing the python code. Initially it connects to the MQTT server as follows.
\begin{lstlisting}[language=Python, caption=Python MQTT setup, label={lst:rpmqtt}]
    mqttc = mqtt.Client()
    # Assign event callbacks
    mqttc.on_message = on_message
    mqttc.on_connect = on_connect
    mqttc.on_publish = on_publish
    mqttc.on_subscribe = on_subscribe
    ...
    # Connect
    mqttc.username_pw_set(username, password)
    mqttc.connect(host, port) 
\end{lstlisting}
Where the callbacks are self written functions, and username and password, comes from a config file.\\
The Pi then checks up against its digital twin, to check if its own configuration as fallen behind, while offline.
\begin{lstlisting}[language=Python, caption=Python fetch digital twin config, label={lst:rpdtc}]
    resp = requests.get(apiEndpoint)
    respObj = resp.json()
\end{lstlisting}
The Pi then starts performing its intended task, monitoring the environment. To do so the Python library SenseHat is used. The singleReading functions takes a single reading from all the used sensors and returns them. The readPeriodically takes a periode in seconds and a handler, it then uses the singleReading and passes the readings to the handler, then is uses threading to call it self again after the periode has passed.
\begin{lstlisting}[language=Python, caption=Python reading from sensehat, label={lst:rprfsh}]
    sense = SenseHat()

    def singleReading():
        return [sense.temp, sense.humidity, sense.pressure]
    
    def readPeriodically(periode, handler): 
        handler(singleReading())
        threading.Timer(periode, readPeriodically, [periode, handler]).start()
\end{lstlisting}
\todo[inline]{References!}