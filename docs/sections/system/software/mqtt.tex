\subsection{MQTT Broker}

The MQTT broker is hosted at CloudMQTT. It is easy to sign up with your google account and get started. CloudMQTT has a free plan supporting up to 5 devices. The CloudMQTT portal provides the required API Key and additional properties needed to establish authenticated connections. Using CloudMQTT as message broker is straight forward, there is minimal configuration as the topics are string based and can be published and subscribed to dynamically. However predefined interfaces, in this case, topics is a good practice. Well-defined interfaces allow parallel working and a more neat integration process. Table \ref{tbl:topics} presents all the topics defined in this project and a description from the publisher's perspective.

\begin{table}[H]
    \centering
    \begin{tabular}{|l|p{10cm}|}
    \hline
    \textbf{MQTT Topics}    & \textbf{Description} \\ \hline
    Telemetry & Devices publish acquired sensor telemetry to this topic. \\ \hline
    device/:DeviceId & The server publish update request messages of threshold values to this topic. The topic is filtered by device id. \\ \hline
    Report & Devices publish reports on update request to this topic. \\ \hline
    Alarm & Devices publish alarms when thresholds are exceeded to this topic. \\ \hline
    
    \end{tabular}
    \caption{MQTT Topics}
    \label{tbl:topics}
\end{table}
