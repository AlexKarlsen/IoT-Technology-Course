\subsection{MQTT}

The MQTT broker is hosted at CloudMQTT. The topic-based Publish/Subscribe pattern makes it easy to establish bi-directional communication. To set it up the node module 'mqtt' must be installed. After a MQTT Client is created, we can subscribe to the relevant topics. All we must do is to implement the required callback functions. We can publish simply by calling the publish method and pass the topic and message to send as parameters. Table \ref{tbl:topics} presents all the topics and a description.

\begin{table}[H]
    \centering
    \begin{tabular}{|l|p{10cm}|}
    \hline
    \textbf{MQTT Topics}    & \textbf{Description} \\ \hline
    Telemetry & Devices publish sensor telemetry to this topic. The server subscribes, sends the telemetry REST Endpoint Stream* and stores the value in the database \\ \hline
    Report & Devices publishes reports on update request to this topic. The server send the report to services Rest Endpoint Stream* and updates the device twin \\ \hline
    Alarm & Devices publish alarm when thresholds are exceeded to this topic. The server sends the alarm to the service Rest Endpoint Stream* \\ \hline
    device/:DeviceId & The server publish update request messages to this filtered device topic. The device updates its settings and report back to the server. \\ \hline
    \end{tabular}
    \caption{MQTT Topics.\\ * REST Enpoint are explained in table \ref{tbl:device}, \ref{tbl:telemetry} and \ref{tbl:service}}
    \label{tbl:topics}
\end{table}
