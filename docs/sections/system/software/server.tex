\subsection{Server}
The server is the entity that binds the solution together. It manges all data of the solution by communicating with the database. It manages communication with the devices trough MQTT. Lastly it serves as a REST API for the device to retrieve its' setting and to the application to get telemetry data and manage devices. The server is hosted at Heroku \cite{heroku}. 

\subsubsection{MQTT}
The server provides MQTT capabilities by installing the MQTT NPM package \cite{mqtt}. The API provides an mqtt client object with callback functions for connecting, subscribing, publishing and to receive messages. These methods needs to be implement in the same fashion as the python library for the device. Listing \ref{lst:nodemqtt} shows how to connect and subscribe and it shows th callback handles upon connection or when receiving messages. 

\begin{lstlisting}[language=Python, caption=Node.js MQTT, label={lst:nodemqtt}, basicstyle=\tiny]
    // Connect mqtt with credentials
    this.mqttClient = mqtt.connect(this.host, this.options);

    // Connection callback
    this.mqttClient.on('connect', () => {
      console.log(`mqtt client connected`);
    });

    // Subscribe to required subscriptions
    this.mqttClient.subscribe(subscriptions, { qos: 0 });

    // When a message arrives handle it accordingly
    this.mqttClient.on('message', function (topic, message) {
        ...
    }
\end{lstlisting}

\subsubsection{REST}
The server exposes services to the device and the application in a RESTful architectural manner. The Rest API exposes one service to devices, where a device can get it's own setting. This service is used upon start-up to get updates, that happened while the device was offline. 
The REST API exposes also services to get all telemetry data, and to get telemetry data filtered for a single device and to establish a live stream of device data. 
The REST API exposes services to the application. Services as; get all the devices, get a device's settings, update a device thresholds and get live stream of device events. The data are stored in a Mongo DB, the server uses the mongodb driver for node.js \cite{mongodriver} to implement the needed services. The exposed HTTP endpoints can be found in API tables after the tables the services of the API are elaborated upon. Table \ref{tbl:device} presents the device endpoints. 

\begin{table}[H]
    \centering
    \begin{tabular}{|l|l|l|}
    \hline
    \textbf{HTTP Method}    & \textbf{API Route} & \textbf{Response} \\ \hline
    Get & /Device/:id & Device's settings  \\ \hline
    \end{tabular}
    \caption{REST Endpoint Device}
    \label{tbl:device}
\end{table}
The server also exposes HTTP REST Endpoint for applications to get telemetry data from devices as seen in table \ref{tbl:telemetry}.

\begin{table}[H]
    \centering
    \begin{tabular}{|l|l|p{5cm}|}
    \hline
    \textbf{HTTP Method}    & \textbf{API Route} & \textbf{Response} \\ \hline
    Get & /Telemetry & All Sensor Telemetry  \\ \hline
    Get & /Telemetry/:id & Sensor Telemetry by device \\ \hline
    Stream & /Telemetry/stream & Live SSE Sensor Telemetry  \\ \hline
    \end{tabular}
    \caption{REST Endpoint Telemetry}
    \label{tbl:telemetry}
\end{table}
The server exposes HTTP REST endpoint service to manage devices and subscribe to the reports and alerts stream as seen in table \ref{tbl:service}.

    \begin{table}[H]
        \centering
        \begin{tabular}{|l|l|p{5cm}|}
        \hline
        \textbf{HTTP Method}    & \textbf{API Route} & \textbf{Response} \\ \hline
        Get & /Service/Device & List of all devices  \\ \hline
        Get & /Service/Device:id & Device's Setting \\ \hline
        Put & /Service/Device/:id/Threshold/:type & Update Threshold Value for device by type  \\ \hline
        Stream & /Service/Device/Stream & Live SSE Reports and Alerts \\ \hline
        \end{tabular}
        \caption{REST Endpoint Service}
        \label{tbl:service}
    \end{table}

    Node.js Express Framework provides a way of nicely constructing the API routes and separate the logic into controllers in a MVC-like pattern returning JSON data instead of rendered views. File api/routes/serviceRoutes.js presented in listing \ref{lst:routes} show the routing structure of the REST API Service Endpoint. 

\begin{lstlisting}[language=Python, caption=Node.js REST API Routes, label={lst:routes}, basicstyle=\tiny]
    ...
    var baseurl = '/service';

    /* REST API routes */
    router.route(baseurl + '/device')
        .get(serviceController.getDevices);

    router.route(baseurl + '/device/:id')
        .get(serviceController.getDeviceSetting);

    router.route(baseurl + '/device/:id' + '/thresholds' + '/:type')
        .put(serviceController.updateDeviceSetting);

    router.route(baseurl + '/stream')
        .get(serviceController.deviceStream); 
    ...
\end{lstlisting}

\subsubsection{Stream Data}
The server enables live stream features by implementing the Server-Sent-Event (SSE) pattern. SSE is an altered HTTP Get Method, where the connection is kept alive. The server sends events event over this open connection to applications, that utilizes this channel and live updates the UI. Below listing \ref{lst:sse} shows how it is implemented.

\begin{lstlisting}[language=Python, caption=Server Sent Event, label={lst:sse}, basicstyle=\tiny]
    router.get('/api/telemetry/stream', function (req, res) {
    res.writeHead(200, {
      'Content-Type': 'text/event-stream',
      'Cache-Control': 'no-cache',
      'Connection': 'keep-alive'
    });

    var topic = "telemetry";
    client.subscribe(topic, function () {
      client.on('message', function (topic, msg, pkt) {
        res.write("data: " + msg + '\n\n');
      });
    });
  });
});
\end{lstlisting}

