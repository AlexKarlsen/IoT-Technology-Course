\subsection{Server}
The server is the entity that binds the solution together. It serves as a REST API to the device and the application. It communicates with the devices trough MQTT protocol and sits in front of the database responsible for reading and writing data from the IoT device and the Apllication. \\\\ The server handles communication with the IoT Device through the MQTT broker. The MQTT broker is hosted at CloudMQTT. The topic-based Publish/Subscribe pattern makes it easy to establish bi-directional communication. To set it up the node module 'mqtt' must be installed. After a MQTT Client is created, we can subscribe to the relevant topics. All we must do is to implement the required callback functions. We can publish simply by calling the publish method and pass the topic and message to send as parameters. Table \ref{tbl:topics} presents all the topics and a description.

\begin{table}[H]
    \centering
    \begin{tabular}{|l|p{10cm}|}
    \hline
    \textbf{MQTT Topics}    & \textbf{Description} \\ \hline
    Telemetry & Devices publish sensor telemetry to this topic. The server subscribes, sends the telemetry REST Endpoint Stream* and stores the value in the database \\ \hline
    Report & Devices publishes reports on update request to this topic. The server send the report to services Rest Endpoint Stream* and updates the device twin \\ \hline
    Alarm & Devices publish alarm when thresholds are exceeded to this topic. The server sends the alarm to the service Rest Endpoint Stream* \\ \hline
    device/:DeviceId & The server publish update request messages to this filtered device topic. The device updates its settings and report back to the server. \\ \hline
    \end{tabular}
    \caption{MQTT Topics.\\ * REST Enpoint are explained in table \ref{tbl:device}, \ref{tbl:telemetry} and \ref{tbl:service}}
    \label{tbl:topics}
\end{table}

The server also serves as a REST API exposing services to the device and the application. The REST API exposes a service to the device to get it's own setting. This service is used upon start-up to get updates, that happened while the device was offline. 
The REST API exposes also services to get all telemetry data, and to get telemetry data filtered for a single device and to establish a live stream of device data. 
The REST API exposes services to the application. Services as; to get all the devices, to get a device's settings,to update a device thresholds and to get live stream of device events. The exposed HTTP endpoints can be found in API tables after the tables the services of the API are elaborated upon. Table \ref{tbl:device} presents the device endpoints. 

\begin{table}[H]
    \centering
    \begin{tabular}{|l|l|l|}
    \hline
    \textbf{HTTP Method}    & \textbf{API Route} & \textbf{Response} \\ \hline
    Get & /Device/:id & Device's settings  \\ \hline
    \end{tabular}
    \caption{REST Endpoint Device}
    \label{tbl:device}
\end{table}
The server also exposes HTTP REST Endpoint for applications to get telemetry data from devices as seen in table \ref{tbl:telemetry}.

\begin{table}[H]
    \centering
    \begin{tabular}{|l|l|p{5cm}|}
    \hline
    \textbf{HTTP Method}    & \textbf{API Route} & \textbf{Response} \\ \hline
    Get & /Telemetry & All Sensor Telemetry  \\ \hline
    Get & /Telemetry/:id & Sensor Telemetry by device \\ \hline
    Stream & /Telemetry/stream & Live SSE Sensor Telemetry  \\ \hline
    \end{tabular}
    \caption{REST Endpoint Telemetry}
    \label{tbl:telemetry}
\end{table}
The server exposes HTTP REST endpoint service to manage devices and subscribe to the reports and alerts stream as seen in table \ref{tbl:service}.

    \begin{table}[H]
        \centering
        \begin{tabular}{|l|l|p{5cm}|}
        \hline
        \textbf{HTTP Method}    & \textbf{API Route} & \textbf{Response} \\ \hline
        Get & /Service/Device & List of all devices  \\ \hline
        Get & /Service/Device:id & Device's Setting \\ \hline
        Put & /Service/Device/:id/Threshold/:type & Update Threshold Value for device by type  \\ \hline
        Stream & /Service/Device/Stream & Live SSE Reports and Alerts \\ \hline
        \end{tabular}
        \caption{REST Endpoint Service}
        \label{tbl:service}
    \end{table}

    Node.js Express Framework provides a way of nicely constructing the API routes and separate the logic into controllers in a MVC-like pattern returning JSON data instead of rendered views. File api/routes/serviceRoutes.js presented in listing \ref{lst:routes} show the routing structure of the REST API Service Endpoint. 

\begin{lstlisting}[style=js, caption=Node.js REST API Routes, label={lst:routes}]
    ...
    var baseurl = '/service';

    /* REST API routes */
    router.route(baseurl + '/device')
        .get(serviceController.getDevices);

    router.route(baseurl + '/device/:id')
        .get(serviceController.getDeviceSetting);

    router.route(baseurl + '/device/:id' + '/thresholds' + '/:type')
        .put(serviceController.updateDeviceSetting);

    router.route(baseurl + '/stream')
        .get(serviceController.deviceStream); 
    ...
\end{lstlisting}

The server enables live stream features by implementing the Server-Sent-Event (SSE) pattern. SSE is an altered HTTP Get Method, where the connection is kept alive. The server sends events event over this open connection to applications, that utilizes this channel and live updates the UI. Below listing \ref{lst:sse} shows how it is implemented.

\begin{lstlisting}[style=js, caption=Server Sent Event, label={lst:sse}]
    router.get('/api/telemetry/stream', function (req, res) {
    // send headers for event-stream connection
    res.writeHead(200, {
      'Content-Type': 'text/event-stream',
      'Cache-Control': 'no-cache',
      'Connection': 'keep-alive'
    });

    var topic = "telemetry";
    client.subscribe(topic, function () {
      client.on('message', function (topic, msg, pkt) {
        res.write("data: " + msg + '\n\n');
      });
    });
  });
});
\end{lstlisting}

