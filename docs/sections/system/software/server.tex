\subsection{Server}
The server is the entity that binds the solution together. It is implemented as a Node.js server written in javascript.  The server handles all communication with the IoT Device by an MQTT broker. It handles all communication with the client applications and enable real-time feature by implementing Server-Sent-Event (SSE). \todo{elaborate} Lastly it handles all interaction with the database.   
Almost all communication with the device is done trough the MQTT broker hosted at CloudMQTT. The topic-based Publish/Subscribe pattern makes it easy to establish bi-directional communication. To set it app the node module 'mqtt' must be installed. It allows us to create a MQTT Client and to subscribe all we must do is subscribe to the defined topics and implement the requires callback functions. To publish simply call the publish function and pass on the topic and message to send. Table \ref{tbl:topics} presents all the topics and a description.

\begin{table}[H]
    \centering
    \begin{tabular}{|l|p{10cm}|}
    \hline
    \textbf{MQTT Topics}    & \textbf{Description} \\ \hline
    Telemetry & Devices publish sensor telemetry to this topic. The server subscribes, sends the telemetry REST Endpoint Stream* and stores the value in the database \\ \hline
    Report & Devices publishes reports on update request to this topic. The server send the report to services Rest Endpoint Stream* and updates the device twin \\ \hline
    Alarm & Devices publish alarm when thresholds are exceeded to this topic. The server sends the alarm to the service Rest Endpoint Stream* \\ \hline
    device/:DeviceId & The server publish update request messages to this filtered device topic. The device updates its settings and report back to the server. \\ \hline
    \end{tabular}
    \caption{MQTT Topics.\\ * REST Enpoint are explained in table \ref{tbl:device}, \ref{tbl:telemetry} and \ref{tbl:service}}
    \label{tbl:topics}
\end{table}
The server exposes an HTTP REST endpoint to the device to get the device's device twin, as seen in API table \ref{tbl:device}.

\begin{table}[H]
    \centering
    \begin{tabular}{|l|l|l|}
    \hline
    \textbf{HTTP Method}    & \textbf{API Route} & \textbf{Response} \\ \hline
    Get & /Device/:id & Get Device's settings  \\ \hline
    \end{tabular}
    \caption{REST Endpoint Device}
    \label{tbl:device}
\end{table}
The server also exposes HTTP REST Endpoint for applications to get telemetry data from devices as seen in table \ref{tbl:telemetry}.

\begin{table}[H]
    \centering
    \begin{tabular}{|l|l|p{5cm}|}
    \hline
    \textbf{HTTP Method}    & \textbf{API Route} & \textbf{Response} \\ \hline
    Get & /Telemetry & All Sensor Telemetry  \\ \hline
    Get & /Telemetry/:id & Sensor Telemetry by device \\ \hline
    Stream & /Telemetry/stream & Live SSE Sensor Telemetry  \\ \hline
    \end{tabular}
    \caption{REST Endpoint Telemetry}
    \label{tbl:telemetry}
\end{table}
The server exposes HTTP REST endpoint service to manage devices and subscribe to the reports and alerts stream as seen in table \ref{tbl:service}.

    \begin{table}[H]
        \centering
        \begin{tabular}{|l|l|p{5cm}|}
        \hline
        \textbf{HTTP Method}    & \textbf{API Route} & \textbf{Response} \\ \hline
        Get & /Service/Device & Get All Devices  \\ \hline
        Get & /Service/Device:id & Get Device Setting \\ \hline
        Put & /Service/Device/:id/Threshold/:type & Update Threshold Value for device by type  \\ \hline
        Stream & /Service/Device/Stream & Live SSE Reports and Alerts \\ \hline
        \end{tabular}
        \caption{REST Endpoint Service}
        \label{tbl:service}
    \end{table}

    File api/routes/serviceRoutes.js presented in listing \ref{lst:routes} show the routing structure of the REST API Service Endpoint. 

\begin{lstlisting}[style=js, caption=Node.js REST API Routes, label={lst:routes}]
    ...
    var baseurl = "/service";

    /* REST API routes */
    router.route(baseurl + '/device')
        .get(serviceController.getDevices);

    router.route(baseurl + '/device/:id')
        .get(serviceController.getDeviceSetting);

    router.route(baseurl + '/device/:id' + '/thresholds' + '/:type')
        .put(serviceController.updateDeviceSetting);

    router.route(baseurl + '/stream')
        .get(serviceController.deviceStream); 
    ...
\end{lstlisting}

The server implements SSE to enable real-time features. SSE is an altered HTTP Get Method, where the connection is kept alive. The server sends events event over this open connection to applications, that utilizes this channel and live updates the UI.

Insert code example of SSE 

