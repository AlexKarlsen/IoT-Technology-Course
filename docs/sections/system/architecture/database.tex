The database purpose is to store telemetry data and device twins. NoSQL databases like MongoDb \cite{mongo} is a good match. NoSQL stores data as JSON documents rather than relational tables. NoSQL overcomes the issue of having SQL database on clusters, which allow big data storage. An important aspect as the amount of data collected by IOT device are expected to grow rapidly. Already in most IoT systems the data storage requires more capacity, than a single computer can provide. NoSQL also overcomes the impedance mismatch between in-memory data types and the data stored in relational database \cite{nosql}. NoSQl store documents in for instance the JSON format, that is a well suited format because of the aforementioned compactness required for constrained IoT devices. Sensor readings can be stored as they are sent from the device by appending to the document and device twin can be stored in the database in the exact same format as on the device and updating the fields when required. NoSQL requires no schema, which means the structure can easily be altered, ideally for rapid prototyping than the more fixed schema approach of SQL.